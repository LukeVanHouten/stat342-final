% Options for packages loaded elsewhere
\PassOptionsToPackage{unicode}{hyperref}
\PassOptionsToPackage{hyphens}{url}
%
\documentclass[
  11pt,
]{article}
\usepackage{amsmath,amssymb}
\usepackage{iftex}
\ifPDFTeX
  \usepackage[T1]{fontenc}
  \usepackage[utf8]{inputenc}
  \usepackage{textcomp} % provide euro and other symbols
\else % if luatex or xetex
  \usepackage{unicode-math} % this also loads fontspec
  \defaultfontfeatures{Scale=MatchLowercase}
  \defaultfontfeatures[\rmfamily]{Ligatures=TeX,Scale=1}
\fi
\usepackage{lmodern}
\ifPDFTeX\else
  % xetex/luatex font selection
\fi
% Use upquote if available, for straight quotes in verbatim environments
\IfFileExists{upquote.sty}{\usepackage{upquote}}{}
\IfFileExists{microtype.sty}{% use microtype if available
  \usepackage[]{microtype}
  \UseMicrotypeSet[protrusion]{basicmath} % disable protrusion for tt fonts
}{}
\makeatletter
\@ifundefined{KOMAClassName}{% if non-KOMA class
  \IfFileExists{parskip.sty}{%
    \usepackage{parskip}
  }{% else
    \setlength{\parindent}{0pt}
    \setlength{\parskip}{6pt plus 2pt minus 1pt}}
}{% if KOMA class
  \KOMAoptions{parskip=half}}
\makeatother
\usepackage{xcolor}
\usepackage[margin=1in]{geometry}
\usepackage{color}
\usepackage{fancyvrb}
\newcommand{\VerbBar}{|}
\newcommand{\VERB}{\Verb[commandchars=\\\{\}]}
\DefineVerbatimEnvironment{Highlighting}{Verbatim}{commandchars=\\\{\}}
% Add ',fontsize=\small' for more characters per line
\usepackage{framed}
\definecolor{shadecolor}{RGB}{248,248,248}
\newenvironment{Shaded}{\begin{snugshade}}{\end{snugshade}}
\newcommand{\AlertTok}[1]{\textcolor[rgb]{0.94,0.16,0.16}{#1}}
\newcommand{\AnnotationTok}[1]{\textcolor[rgb]{0.56,0.35,0.01}{\textbf{\textit{#1}}}}
\newcommand{\AttributeTok}[1]{\textcolor[rgb]{0.13,0.29,0.53}{#1}}
\newcommand{\BaseNTok}[1]{\textcolor[rgb]{0.00,0.00,0.81}{#1}}
\newcommand{\BuiltInTok}[1]{#1}
\newcommand{\CharTok}[1]{\textcolor[rgb]{0.31,0.60,0.02}{#1}}
\newcommand{\CommentTok}[1]{\textcolor[rgb]{0.56,0.35,0.01}{\textit{#1}}}
\newcommand{\CommentVarTok}[1]{\textcolor[rgb]{0.56,0.35,0.01}{\textbf{\textit{#1}}}}
\newcommand{\ConstantTok}[1]{\textcolor[rgb]{0.56,0.35,0.01}{#1}}
\newcommand{\ControlFlowTok}[1]{\textcolor[rgb]{0.13,0.29,0.53}{\textbf{#1}}}
\newcommand{\DataTypeTok}[1]{\textcolor[rgb]{0.13,0.29,0.53}{#1}}
\newcommand{\DecValTok}[1]{\textcolor[rgb]{0.00,0.00,0.81}{#1}}
\newcommand{\DocumentationTok}[1]{\textcolor[rgb]{0.56,0.35,0.01}{\textbf{\textit{#1}}}}
\newcommand{\ErrorTok}[1]{\textcolor[rgb]{0.64,0.00,0.00}{\textbf{#1}}}
\newcommand{\ExtensionTok}[1]{#1}
\newcommand{\FloatTok}[1]{\textcolor[rgb]{0.00,0.00,0.81}{#1}}
\newcommand{\FunctionTok}[1]{\textcolor[rgb]{0.13,0.29,0.53}{\textbf{#1}}}
\newcommand{\ImportTok}[1]{#1}
\newcommand{\InformationTok}[1]{\textcolor[rgb]{0.56,0.35,0.01}{\textbf{\textit{#1}}}}
\newcommand{\KeywordTok}[1]{\textcolor[rgb]{0.13,0.29,0.53}{\textbf{#1}}}
\newcommand{\NormalTok}[1]{#1}
\newcommand{\OperatorTok}[1]{\textcolor[rgb]{0.81,0.36,0.00}{\textbf{#1}}}
\newcommand{\OtherTok}[1]{\textcolor[rgb]{0.56,0.35,0.01}{#1}}
\newcommand{\PreprocessorTok}[1]{\textcolor[rgb]{0.56,0.35,0.01}{\textit{#1}}}
\newcommand{\RegionMarkerTok}[1]{#1}
\newcommand{\SpecialCharTok}[1]{\textcolor[rgb]{0.81,0.36,0.00}{\textbf{#1}}}
\newcommand{\SpecialStringTok}[1]{\textcolor[rgb]{0.31,0.60,0.02}{#1}}
\newcommand{\StringTok}[1]{\textcolor[rgb]{0.31,0.60,0.02}{#1}}
\newcommand{\VariableTok}[1]{\textcolor[rgb]{0.00,0.00,0.00}{#1}}
\newcommand{\VerbatimStringTok}[1]{\textcolor[rgb]{0.31,0.60,0.02}{#1}}
\newcommand{\WarningTok}[1]{\textcolor[rgb]{0.56,0.35,0.01}{\textbf{\textit{#1}}}}
\usepackage{graphicx}
\makeatletter
\def\maxwidth{\ifdim\Gin@nat@width>\linewidth\linewidth\else\Gin@nat@width\fi}
\def\maxheight{\ifdim\Gin@nat@height>\textheight\textheight\else\Gin@nat@height\fi}
\makeatother
% Scale images if necessary, so that they will not overflow the page
% margins by default, and it is still possible to overwrite the defaults
% using explicit options in \includegraphics[width, height, ...]{}
\setkeys{Gin}{width=\maxwidth,height=\maxheight,keepaspectratio}
% Set default figure placement to htbp
\makeatletter
\def\fps@figure{htbp}
\makeatother
\setlength{\emergencystretch}{3em} % prevent overfull lines
\providecommand{\tightlist}{%
  \setlength{\itemsep}{0pt}\setlength{\parskip}{0pt}}
\setcounter{secnumdepth}{-\maxdimen} % remove section numbering
\newlength{\cslhangindent}
\setlength{\cslhangindent}{1.5em}
\newlength{\csllabelwidth}
\setlength{\csllabelwidth}{3em}
\newlength{\cslentryspacingunit} % times entry-spacing
\setlength{\cslentryspacingunit}{\parskip}
\newenvironment{CSLReferences}[2] % #1 hanging-ident, #2 entry spacing
 {% don't indent paragraphs
  \setlength{\parindent}{0pt}
  % turn on hanging indent if param 1 is 1
  \ifodd #1
  \let\oldpar\par
  \def\par{\hangindent=\cslhangindent\oldpar}
  \fi
  % set entry spacing
  \setlength{\parskip}{#2\cslentryspacingunit}
 }%
 {}
\usepackage{calc}
\newcommand{\CSLBlock}[1]{#1\hfill\break}
\newcommand{\CSLLeftMargin}[1]{\parbox[t]{\csllabelwidth}{#1}}
\newcommand{\CSLRightInline}[1]{\parbox[t]{\linewidth - \csllabelwidth}{#1}\break}
\newcommand{\CSLIndent}[1]{\hspace{\cslhangindent}#1}
\usepackage{amsmath}
\usepackage{amsthm}
\ifLuaTeX
  \usepackage{selnolig}  % disable illegal ligatures
\fi
\IfFileExists{bookmark.sty}{\usepackage{bookmark}}{\usepackage{hyperref}}
\IfFileExists{xurl.sty}{\usepackage{xurl}}{} % add URL line breaks if available
\urlstyle{same}
\hypersetup{
  pdftitle={Placebo-Controlled Efficacy Analysis of the COVID-19 Vaccine},
  pdfauthor={Oliver Brown, Josie Czeskleba, Luke VanHouten},
  hidelinks,
  pdfcreator={LaTeX via pandoc}}

\title{Placebo-Controlled Efficacy Analysis of the COVID-19 Vaccine}
\usepackage{etoolbox}
\makeatletter
\providecommand{\subtitle}[1]{% add subtitle to \maketitle
  \apptocmd{\@title}{\par {\large #1 \par}}{}{}
}
\makeatother
\subtitle{Spring 2024}
\author{Oliver Brown, Josie Czeskleba, Luke VanHouten}
\date{}

\begin{document}
\maketitle

\hypertarget{abstract}{%
\section{Abstract}\label{abstract}}

The COVID-19 pandemic, necessitated the development of effective
vaccines to mitigate severe health outcomes. This study evaluates the
efficacy of the Pfizer-BioNTech BNT162b2 mRNA vaccine, using statistical
methods such as maximum likelihood estimation (MLE), likelihood ratio
tests, and bootstrapping. Analyzing data from a placebo-controlled study
with 170 COVID-19 cases, we estimated the vaccine efficacy to be
approximately 95\%, significantly surpassing the FDA's minimum efficacy
requirement of 30\%. The MLE provided an efficacy estimate of 0.9506,
with a 95\% confidence interval ranging from 91.55\% to 98.57\%.
Bootstrapping with 10,000 iterations confirmed the robustness of this
estimate, yielding similar results. Our findings, demonstrate the
vaccine's high effectiveness in preventing COVID-19.

\hypertarget{keywords}{%
\section{Keywords}\label{keywords}}

Efficacy, Likelihood, COVID-19, Bootstrap, Estimators

\hypertarget{introduction}{%
\section{Introduction}\label{introduction}}

The COVID-19 pandemic has impacted millions globally, since the World
Health Organization declared it a global pandemic on March 11th, 2020.
Sensitive groups are at high risk of COVID-19 and its complications.
Moreover, recent studies have shown that long-term COVID-19 or
post-COVID-19 symptoms make individuals more susceptible to additional
health conditions like diabetes, heart conditions, and neurological
conditions (CDC). Therefore a safe and effective vaccine is imperative
to global health.

In December 2020, Pfizer and BioNTech received FDA Emergency Use
Authorization for their two-dose BNT162b2 COVID-19 mRNA vaccine. The
approval was based on a placebo-controlled, observer-blinded study
involving participants aged 16 and older who were randomly assigned to
receive either the placebo or the vaccine. The data provided leverage to
prove that the BNT162b2 vaccine efficacy exceeds the FDA-required
efficacy of 30\%.(Polack et al. 2020).

Given the FDA's minimum efficacy requirement of 30\%, this paper focuses
on the statistical analysis of the vaccine's efficacy. We will utilize
hypothesis testing, bootstrapping, and maximum likelihood estimation to
evaluate the data. The thoroughness of this analysis is essential for
ensuring reliable and valid findings, which will guide public health
decisions based on robust evidence.

\hypertarget{statistical-methods}{%
\section{Statistical Methods}\label{statistical-methods}}

The data for positive and negative tests between the placebo and vaccine
groups is taken from the previous research on the vaccine efficacy
(Polack et al. 2020):

\begin{table}[h]
    \centering
    \begin{tabular}{c|p{1in}|p{1in}|p{1in}}
    Test & Positive & Negative & Total \\ \hline
    Vaccine & 8 & 17403 & 17411 \\ \hline
    Placebo & 162 & 17349 & 17511 \\ \hline
    Total & 170 & 34752 & 34922 \\ \hline
    \end{tabular}
\end{table}

\newpage

While this is a rather simple dataset, we can use a stacked bar plot to
better visualize it:

\includegraphics{Writeup_files/figure-latex/stacked barplot-1.pdf}

We denote the random variable \(T\) as the number of vaccinated
individuals from the 170 COVID cases.

\[ T \sim Binom(n = 170,\pi) \]

We can define
\(\pi = \textrm{P(Vaccine|COVID)} = \frac{\pi_1}{\pi_1 + \pi_2}\), given
that the sample sizes for the vaccine and placebo groups are
approximately equal. Here, \(\pi_1\) is the proportion of vaccinated
individuals who got COVID and \(\pi_2\) is the proportion of
unvaccinated individuals who got COVID. Moreover, we define the vaccine
efficacy as \(\psi = \frac{1-2\pi}{1-\pi}\) (Senn 2021). We can then
formulate the following hypothesis test based on the FDA's suggested
efficacy of at least 30\%:

\[ H_0: \psi_0 = 0.3 \] \[ H_1: \psi_0 \geq 0.3 \]

We will use a maximum likelihood estimator to conduct a likelihood ratio
test as well as bootstrapped confidence intervals to test our
hypothesis.

\hypertarget{maximum-likelihood-estimator}{%
\subsection{Maximum Likelihood
Estimator}\label{maximum-likelihood-estimator}}

The first method we used to analyze the BNT162b2 vaccine efficacy is
maximum likelihood estimation. We found the maximum likelihood estimate
for our efficacy parameter \(\psi\), which we can call
\(\hat{\psi}^{mle}_0\). To calculate this estimator, we can first write
the likelihood function of \(\pi\) based on the PDF of \(T\):

\[ L(\pi) = \binom{n}{t}\pi^t(1 - \pi)^{n - t} \]

Then we write \(\pi\) in the form \(\pi = g(\psi)\), given that
\(\psi = \frac{1 - 2\pi}{1 - \pi}\). We thus have that
\(\psi - \psi\pi = 1 - 2\pi\), which becomes
\(2\pi - \psi\pi = 1 - \psi\), which becomes:

\[ \pi = \frac{1 - \psi}{2 - \psi} \]

Because \(\psi\) is the parameter we want to make our estimator,we can
rewrite our likelihood function in terms of \(\psi\), still based on the
PDF of \(T\) for \(t_{obs} = 8\):

\[ L(\psi) = L(g(\psi)) = L\left(\frac{1 - \psi}{2 - \psi}\right) = \binom{n}{t}\left(\frac{1 - \psi}{2 - \psi}\right)^t\left(1 - \left(\frac{1-\psi}{2-\psi}\right)\right)^{n - t} = \binom{n}{t}\left(\frac{1 - \psi}{2 - \psi}\right)^t\left(\frac{1}{2 - \psi}\right)^{n - t} \]

Mathematically, it is difficult to do much with just this function, so
we can log-transform it to convert it from a product to a sum. We can
calculate the log-likelihood function for \(\psi\) as shown here:

\[ \ell(\psi) = \ln(L(\psi) = \ln\left(\binom{n}{t}\right) + t\ln(1 - \psi) - t\ln(2 - \psi) - (n - t)\ln(2 - \psi) = \ln\left(\binom{n}{t}\right) + t\ln(1 - \psi) - n\ln(2 - \psi) \]

Our estimator is defined as where the log-likelihood function is
maximized by the parameter \(\psi\), and we can find this by computing
\(\underset{\psi}{\arg\max} \ \ell(\psi | t)\), or in other words, by
taking the derivative of this function and setting it equal to 0. So we
can calculate \(\ell'(\psi) = 0\) here:

\[ \frac{d}{d\psi} \ell(\psi) = \frac{d}{d\psi} \ln\left(\binom{n}{t}\right) + \frac{d}{d\psi} t\ln(1 - \psi) - \frac{d}{d\psi} n\ln(2 - \psi) = \frac{n}{2 - \psi} - \frac{t}{1 - \psi} = 0 \]

We can then solve \(\frac{n}{2 - \psi} = \frac{t}{1 - \psi}\). We get
that \(n - n\psi = 2t - t\psi\), which becomes
\(t\psi - n\psi = 2t - n\). Assuming that this critical point is a
maximum, we have that our estimator estimator is
\(\widehat{\psi}^{mle}_0 = \frac{2t - n}{t - n}\). We cannot use the
estimator on its own to make much inference about our hypothesis.
However, e can use our MLE to perform a likelihood ratio test, defined
as
\(W = 2\left(\ell\left(\hat{\psi}^{mle}_0\right) - \ell\left(\psi^{null}_0\right)\right)\).
A small \(W\) corresponds with statistical significance for rejecting
the null hypothesis, and vice versa. For a sufficiently sized \(n\),
which we have as \(n = 170\), the distribution \(W \sim \chi^2_1\)
holds, letting us compute a P-value for our hypothesis. We can compute
this value of \(W\) here as:

\[ W = 2\left(\left(\ln\left(\binom{n}{t}\right) + t\ln\left(1 - \widehat{\psi}^{mle}_0\right) - n\ln\left(2 - \widehat{\psi}^{mle}_0\right)\right) - \left(\ln\left(\binom{n}{t}\right) + t\ln\left(1 - \psi^{null}_0\right) - n\ln\left(2 - \psi^{null}_0\right)\right)\right)\]

This becomes:

\[ W = 2t\ln\left(1 - \widehat{\psi}^{mle}_0\right) - 2n\ln\left(2 - \widehat{\psi}^{mle}_0\right) - 2t\ln\left(1 - \psi^{null}_0\right) + 2n\ln\left(2 - \psi^{null}_0\right) \]

Finally, we can set up an equation to find a 95\% confidence interval
for our parameter \(\psi\):

\[ \widehat{\psi}^{\text{mle}}_0 \pm 1.96 \sqrt{\frac{-1}{\ell''(\widehat{\psi}^{\text{mle}}_0)}} = \widehat{\psi}^{\text{mle}}_0 \pm 1.96 \sqrt{\frac{-1}{\frac{n}{(2 - \widehat{\psi}^{\text{mle}}_0)^2} - \frac{t}{(1 - \widehat{\psi}^{\text{mle}}_0)^2}}} \]

\hypertarget{bootstrap}{%
\subsection{Bootstrap}\label{bootstrap}}

Bootstrapping is a powerful statistical technique used to estimate the
sampling distribution of a statistic by repeatedly resampling with
replacement from the original data. This method allows for the
assessment of the variability and accuracy of the statistic without
relying on strong parametric assumptions. By generating numerous
resampled sets of data, bootstrapping creates a distribution of the
statistic of interest, which can be used to derive confidence intervals.

In our analysis, bootstrapping is applied to estimate the efficacy of
the BNT162b2 vaccine. By generating multiple resampled datasets from the
original data, we can repeatedly calculate the vaccine efficacy and
construct a distribution of these efficacy estimates. This distribution
provides valuable insights into the range, variability, and confidence
intervals of the vaccine's effectiveness. Using bootstrapping, we can
assess the reliability of the observed efficacy, ensuring that the
conclusions drawn from the data are well-supported by a rigorous
statistical framework.

In this analysis, we conducted a bootstrap resampling procedure to
evaluate the distribution of the psi values, a derived statistic from
the proportions of COVID cases in vaccine and placebo groups. The
bootstrap technique involves repeatedly sampling with replacement from
the original data to create numerous simulated samples. This method
helps in estimating the sampling distribution of the statistic of
interest and provides insights into its variability and confidence
intervals.

The proportions of COVID cases in each group \((\texttt{prop_vaccine}\)
and \(\texttt{prop_placebo})\) were calculated using the provided data.

\[ \text{Observed }\pi = \frac{\text{Observed }\pi_1}{\text{Observed }\pi_1 + \text{Observed }\pi_2} \]

Using these proportions, the observed psi value was computed using the
formula:

\[\text{Observed } \psi = \frac{1 - 2 \times \text{Observed } \pi}{1 - \text{Observed } \pi}\]

where \(\text{observed_pi}\) is the proportion of COVID cases in the
vaccine group relative to the combined total of COVID cases in both
groups.

A total of 10,000 bootstrap samples were generated. For each iteration,
random samples were drawn with replacement from the \(\texttt{vaccine}\)
and \(\texttt{placebo}\) groups based on their respective proportions.
The bootstrap psi value was then calculated for each resampled dataset.
This process generated a distribution of psi values, stored in
\(\texttt{bootstrap_df}\).

The summary statistics of the bootstrap distribution, including the
mean, standard deviation, and 95\% confidence intervals, were
calculated. The critical values for the confidence intervals were
determined using the quantiles of the normal distribution:

\[\text{Lower critical value} = E[\psi]- Z_{0.975} \times SD[\psi]\]

\[\text{Upper critical value} = E[\psi] + Z_{0.975} \times SD[\psi]\]

The analysis involved calculating the proportions of COVID cases
(\(\pi\)), deriving psi values from these proportions, and using the
bootstrap method to generate 10,000 samples. This allowed for estimation
of the mean, standard deviation, and 95\% confidence intervals of the
psi values.

\hypertarget{results}{%
\section{Results}\label{results}}

For our MLE, we can plug in \(t_{obs} = 8\) and \(n = 170\) to
\(\widehat{\psi}^{mle}_0 = \frac{2t - n}{t - n}\), we get
\(\widehat{\psi}^{mle}_0 = \frac{16 - 170}{8 - 170} = \frac{77}{81} = 0.9506\).
We can also use the Newton Raphson method to estimate \(\psi\) to get
the same value, shown in the appendix.

For the likelihood ratio test, we can plug in our values
\(\widehat{\psi}^{mle}_0 = 0.9506\), \(\psi^{null}_0 = 0.3\), \(t = 8\),
and \(n = 170\) to our W to get:

\[ W = 2(8)\ln\left(1 - 0.9506\right) - 2(170)\ln\left(2 - 0.9506\right) - 2(8)\ln\left(1 - 0.3\right) + 2(170)\ln\left(2 - 0.3\right) = 121.6012 \]

Given our sample size is large we know \(W \sim \chi^2_1\). Therefore,
we can calculate P-value for our hypothesis as
\(P(W \geq 121.6012) = 0\). This P-value is zero, so we can reject the
null hypothesis under our likelihood ratio test that the COVID-19
vaccine efficacy is 30\%.

Finally, we can use our MLE estimator to calculate a 95\% confidence
interval for \(\psi\):

\[ 0.9506 \pm 1.96\sqrt{\frac{-1}{-3124.317}} = 0.9506 \pm 0.0351\]

We are 95\% confident that the true efficacy of the BNT162b2 vaccine
lies in the interval {[}91.55\%, 98.57\%{]}. This supports our findings
in the likelihood ratio test. There is convincing evidence that the true
efficacy exceeds the FDA-required efficacy of 30\%.

Furthermore, the vaccine efficacy (\(\psi\)) for the BNT162b2 vaccine
was estimated using a bootstrap method with 10,000 iterations, ensuring
accurate and stable estimates. The observed vaccine efficacy from the
original data was calculated using the given formula, resulting in an
estimated \(\psi\) of approximately 95\%.

The bootstrap resampling provided a distribution of \(\psi\) values,
from which the mean efficacy was estimated to be 0.9508 with a standard
deviation of 0.0178. The 95\% confidence interval for the vaccine
efficacy, derived from the bootstrap samples, ranged from approximately
91.59\% to 98.58\%. This confidence interval is quite narrow, indicating
a high level of precision in the efficacy estimate.

The histogram of the bootstrap \(\psi\) values, along with the
confidence intervals, confirms that the estimated efficacy is
consistently around 95\%, supporting the robustness of the estimate.
Furthermore, the probability that the vaccine efficacy exceeds 30\% is
1, with the calculated p-value being effectively zero
(\(3.74 \times 10^{-292}\)). This overwhelming evidence suggests that
the vaccine significantly reduces the risk of infection compared to the
placebo.

In summary, the results from the bootstrap analysis strongly support the
conclusion that the BNT162b2 vaccine is highly effective, with an
estimated efficacy of around 95\%. The narrow confidence interval and
extremely low p-value reinforce the reliability and significance of the
vaccine's protective effect against COVID-19.

\hypertarget{conclusion}{%
\section{Conclusion}\label{conclusion}}

In conclusion, we can say with great confidence that the Pfizer-BioNTech
BNT162b2 COVID-19 vaccine is effective at preventing positive tests of
the disease, as the vaccine efficacy is much greater than the FDA target
value of 30\%. With 95.06\% efficacy (based off of the maxmimum
likelihood estimator for our binomial distribution of positive tests),
this COVID-19 vaccine should be recommend for general use for disease
prevention. We conducted a variety of statistical testing to back up
this claim, such as the likelihood ratio test and boostrapped efficacy
paramters, which both netted extremely small P-values as well as tight
confidence intervals around the estimate for the vaccine efficacy. Our
results match the findings of other research on this vaccine's efficacy,
where Bayesian estimates were used to show that the vaccine is
sufficiently effective (Polack et al. 2020). Next steps for this
research would be to utilize more estimators to reach this conclusion,
such as the method of moments estimator as well as to corroborate the
previous research using Bayesian methods, such as the Beta Binomial
model (Lee and Sabavala 1987). Other next steps would be to use these
methods to identify vaccine efficacy in vaccines for other diseases,
especially newly developed ones.

\hypertarget{references}{%
\section{References}\label{references}}

\hypertarget{refs}{}
\begin{CSLReferences}{1}{0}
\leavevmode\vadjust pre{\hypertarget{ref-longcovid}{}}%
CDC. {``{P}ost-{C}{O}{V}{I}{D} {C}onditions --- Cdc.gov.''}
\url{https://www.cdc.gov/coronavirus/2019-ncov/long-term-effects/index.html}.

\leavevmode\vadjust pre{\hypertarget{ref-BetaBinomial}{}}%
Lee, Jack C., and Darius J. Sabavala. 1987. {``Bayesian Estimation and
Prediction for the Beta-Binomial Model.''} \emph{Journal of Business \&
Economic Statistics} 5 (3): 357--67.
\url{https://doi.org/10.2307/1391611}.

\leavevmode\vadjust pre{\hypertarget{ref-COVIDpaper}{}}%
Polack, Fernando P., Stephen J. Thomas, Nicholas Kitchin, Judith
Absalon, Alejandra Gurtman, Stephen Lockhart, John L. Perez, et al.
2020. {``Safety and Efficacy of the BNT162b2 mRNA Covid-19 Vaccine.''}
\emph{New England Journal of Medicine} 383 (27): 2603--15.
\url{https://doi.org/10.1056/NEJMoa2034577}.

\leavevmode\vadjust pre{\hypertarget{ref-errorstatistics}{}}%
Senn, Stephen. 2021. {``S. {Senn}: {`{Beta} Testing'}: {The}
{Pfizer}/{BioNTech} Statistical Analysis of Their {Covid}-19 Vaccine
Trial (Guest Post).''} \emph{Error Statistics Philosophy}.
\url{https://errorstatistics.com/2021/01/17/s-senn-beta-testing-the-pfizer-biontech-statistical-analysis-of-their-covid-19-vaccine-trial-guest-post/}.

\end{CSLReferences}

\hypertarget{appendix}{%
\section{Appendix}\label{appendix}}

\hypertarget{visualizations}{%
\subsection{Visualizations}\label{visualizations}}

\hypertarget{stacked-barplot-code}{%
\subsubsection{Stacked Barplot Code}\label{stacked-barplot-code}}

\begin{Shaded}
\begin{Highlighting}[]
\FunctionTok{ggplot}\NormalTok{(data\_melted, }\FunctionTok{aes}\NormalTok{(}\AttributeTok{x =}\NormalTok{ Test, }\AttributeTok{y =}\NormalTok{ value, }\AttributeTok{fill =}\NormalTok{ variable)) }\SpecialCharTok{+}
  \FunctionTok{geom\_bar}\NormalTok{(}\AttributeTok{stat =} \StringTok{"identity"}\NormalTok{) }\SpecialCharTok{+}
  \FunctionTok{geom\_text}\NormalTok{(}\FunctionTok{aes}\NormalTok{(}\AttributeTok{label =}\NormalTok{ value), }
            \AttributeTok{position =} \FunctionTok{position\_stack}\NormalTok{(}\AttributeTok{vjust =} \FloatTok{0.5}\NormalTok{)) }\SpecialCharTok{+}
  \FunctionTok{labs}\NormalTok{(}\AttributeTok{x =} \StringTok{"Test"}\NormalTok{, }\AttributeTok{y =} \StringTok{"Number of Participants"}\NormalTok{, }
       \AttributeTok{title =} \StringTok{"COVID Infection for Vaccine and Placebo Groups"}\NormalTok{) }
\end{Highlighting}
\end{Shaded}

\hypertarget{newton-rhapson-mle-approximation}{%
\subsection{Newton Rhapson MLE
Approximation}\label{newton-rhapson-mle-approximation}}

\begin{Shaded}
\begin{Highlighting}[]
\NormalTok{loglik }\OtherTok{\textless{}{-}} \ControlFlowTok{function}\NormalTok{(psi, T, n)\{}
  \ControlFlowTok{if}\NormalTok{ (psi }\SpecialCharTok{\textgreater{}} \DecValTok{1} \SpecialCharTok{|}\NormalTok{ psi }\SpecialCharTok{\textless{}} \DecValTok{0}\NormalTok{) }
    \FunctionTok{return}\NormalTok{(}\ConstantTok{NA}\NormalTok{)}
  \ControlFlowTok{else}
    \FunctionTok{return}\NormalTok{(}\FunctionTok{log}\NormalTok{(}\FunctionTok{choose}\NormalTok{(n, T)) }\SpecialCharTok{+}\NormalTok{ (T }\SpecialCharTok{*} \FunctionTok{log}\NormalTok{(}\DecValTok{1} \SpecialCharTok{{-}}\NormalTok{ psi)) }\SpecialCharTok{{-}}\NormalTok{ (n }\SpecialCharTok{*} \FunctionTok{log}\NormalTok{(}\DecValTok{2} \SpecialCharTok{{-}}\NormalTok{ psi)))}
\NormalTok{\}}

\NormalTok{estimation }\OtherTok{\textless{}{-}} \FunctionTok{maxLik2}\NormalTok{(}\AttributeTok{loglik =}\NormalTok{ loglik, }\AttributeTok{start =} \FloatTok{0.55}\NormalTok{, }\AttributeTok{method =} \StringTok{"NR"}\NormalTok{, }\AttributeTok{tol =} \FloatTok{1e{-}4}\NormalTok{, }
                      \AttributeTok{T =} \DecValTok{8}\NormalTok{, }\AttributeTok{n =} \DecValTok{170}\NormalTok{)}

\FunctionTok{print}\NormalTok{(estimation)}
\end{Highlighting}
\end{Shaded}

\begin{verbatim}
## Maximum Likelihood estimation
## Newton-Raphson maximisation, 6 iterations
## Return code 2: successive function values within tolerance limit (tol)
## Log-Likelihood: -1.944994 (1 free parameter(s))
## Estimate(s): 0.9506174
\end{verbatim}

\begin{Shaded}
\begin{Highlighting}[]
\FunctionTok{plot}\NormalTok{(estimation) }\SpecialCharTok{+}
    \FunctionTok{labs}\NormalTok{(}\AttributeTok{title =} \StringTok{"Second order approximation to the Log{-}Likelihood Function"}\NormalTok{, }
         \AttributeTok{x =} \FunctionTok{expression}\NormalTok{(psi))}
\end{Highlighting}
\end{Shaded}

\includegraphics{Writeup_files/figure-latex/newton rhapson-1.pdf}

\hypertarget{likelihood-ratio-test-calculation}{%
\subsection{Likelihood Ratio Test
Calculation}\label{likelihood-ratio-test-calculation}}

\begin{Shaded}
\begin{Highlighting}[]
\NormalTok{W }\OtherTok{=}\NormalTok{ (}\DecValTok{2} \SpecialCharTok{*} \DecValTok{8} \SpecialCharTok{*} \FunctionTok{log}\NormalTok{(}\DecValTok{1} \SpecialCharTok{{-}}\NormalTok{ (}\DecValTok{77} \SpecialCharTok{/} \DecValTok{81}\NormalTok{))) }\SpecialCharTok{{-}}\NormalTok{ (}\DecValTok{2} \SpecialCharTok{*} \DecValTok{170} \SpecialCharTok{*} \FunctionTok{log}\NormalTok{(}\DecValTok{2} \SpecialCharTok{{-}}\NormalTok{ (}\DecValTok{77} \SpecialCharTok{/} \DecValTok{81}\NormalTok{))) }\SpecialCharTok{{-}}
\NormalTok{    (}\DecValTok{2} \SpecialCharTok{*} \DecValTok{8} \SpecialCharTok{*} \FunctionTok{log}\NormalTok{(}\DecValTok{1} \SpecialCharTok{{-}} \FloatTok{0.3}\NormalTok{)) }\SpecialCharTok{+}\NormalTok{ (}\DecValTok{2} \SpecialCharTok{*} \DecValTok{170} \SpecialCharTok{*} \FunctionTok{log}\NormalTok{(}\DecValTok{2} \SpecialCharTok{{-}} \FloatTok{0.3}\NormalTok{))}

\NormalTok{p\_value }\OtherTok{\textless{}{-}} \FunctionTok{pchisq}\NormalTok{(}\AttributeTok{q =}\NormalTok{ W, }\AttributeTok{df =} \DecValTok{1}\NormalTok{, }\AttributeTok{lower.tail=}\NormalTok{F)}
\end{Highlighting}
\end{Shaded}

Here, \(W\) is 121.6012 and the corresponding P-value is
\ensuremath{2.8222944\times 10^{-28}}.

\hypertarget{mle-confidence-interval-calculation}{%
\subsection{MLE Confidence interval
calculation}\label{mle-confidence-interval-calculation}}

\begin{Shaded}
\begin{Highlighting}[]
\NormalTok{se }\OtherTok{\textless{}{-}} \FunctionTok{sqrt}\NormalTok{(}\SpecialCharTok{{-}}\DecValTok{1} \SpecialCharTok{/}\NormalTok{ ((}\DecValTok{170} \SpecialCharTok{/}\NormalTok{ (}\DecValTok{2} \SpecialCharTok{{-}} \FloatTok{0.9506}\NormalTok{)}\SpecialCharTok{\^{}}\DecValTok{2}\NormalTok{) }\SpecialCharTok{{-}}\NormalTok{ (}\DecValTok{8} \SpecialCharTok{/}\NormalTok{ (}\DecValTok{1} \SpecialCharTok{{-}} \FloatTok{0.9506}\NormalTok{)}\SpecialCharTok{\^{}}\DecValTok{2}\NormalTok{)))}

\NormalTok{upper\_lim }\OtherTok{\textless{}{-}} \FloatTok{0.9506} \SpecialCharTok{+} \FloatTok{1.96} \SpecialCharTok{*}\NormalTok{ se}
\NormalTok{lower\_lim }\OtherTok{\textless{}{-}} \FloatTok{0.9506} \SpecialCharTok{{-}} \FloatTok{1.96} \SpecialCharTok{*}\NormalTok{ se}
\end{Highlighting}
\end{Shaded}

The calculated confidence interval is {[}0.9155, 0.9856{]} with a
standard error of 0.0179.

\hypertarget{bootstrap-1}{%
\subsection{Bootstrap}\label{bootstrap-1}}

\begin{Shaded}
\begin{Highlighting}[]
\NormalTok{data }\OtherTok{\textless{}{-}} \FunctionTok{read\_csv}\NormalTok{(}\StringTok{"data.csv"}\NormalTok{, }\AttributeTok{show\_col\_types =}\NormalTok{ F)}

\FunctionTok{library}\NormalTok{(boot)}
\end{Highlighting}
\end{Shaded}

\begin{verbatim}
## 
## Attaching package: 'boot'
\end{verbatim}

\begin{verbatim}
## The following object is masked from 'package:mosaic':
## 
##     logit
\end{verbatim}

\begin{verbatim}
## The following object is masked from 'package:lattice':
## 
##     melanoma
\end{verbatim}

\begin{Shaded}
\begin{Highlighting}[]
\FunctionTok{library}\NormalTok{(readr)}
\FunctionTok{library}\NormalTok{(tidyverse)}
\end{Highlighting}
\end{Shaded}

\begin{Shaded}
\begin{Highlighting}[]
\NormalTok{vaccine }\OtherTok{\textless{}{-}}\NormalTok{ data }\SpecialCharTok{\%\textgreater{}\%}
  \FunctionTok{filter}\NormalTok{(Test }\SpecialCharTok{==} \StringTok{"Vaccine"}\NormalTok{)}

\NormalTok{placebo }\OtherTok{\textless{}{-}}\NormalTok{ data }\SpecialCharTok{\%\textgreater{}\%}
  \FunctionTok{filter}\NormalTok{(Test }\SpecialCharTok{==} \StringTok{"Placebo"}\NormalTok{)}

\NormalTok{prop\_vaccine }\OtherTok{\textless{}{-}}\NormalTok{ vaccine}\SpecialCharTok{$}\NormalTok{COVID }\SpecialCharTok{/}\NormalTok{ (vaccine}\SpecialCharTok{$}\NormalTok{COVID }\SpecialCharTok{+}\NormalTok{ vaccine}\SpecialCharTok{$}\NormalTok{No\_COVID)}
\NormalTok{prop\_placebo }\OtherTok{\textless{}{-}}\NormalTok{ placebo}\SpecialCharTok{$}\NormalTok{COVID }\SpecialCharTok{/}\NormalTok{ (placebo}\SpecialCharTok{$}\NormalTok{COVID }\SpecialCharTok{+}\NormalTok{ placebo}\SpecialCharTok{$}\NormalTok{No\_COVID)}

\NormalTok{observed\_pi }\OtherTok{\textless{}{-}}\NormalTok{ prop\_vaccine}\SpecialCharTok{/}\NormalTok{(prop\_vaccine }\SpecialCharTok{+}\NormalTok{ prop\_placebo)}

\NormalTok{observed\_psi }\OtherTok{\textless{}{-}}\NormalTok{ (}\DecValTok{1} \SpecialCharTok{{-}} \DecValTok{2}\SpecialCharTok{*}\NormalTok{observed\_pi)}\SpecialCharTok{/}\NormalTok{(}\DecValTok{1} \SpecialCharTok{{-}}\NormalTok{ observed\_pi)}
\end{Highlighting}
\end{Shaded}

\begin{Shaded}
\begin{Highlighting}[]
\NormalTok{n\_bootstrap }\OtherTok{\textless{}{-}} \DecValTok{10000}
\NormalTok{bootstrap\_psis }\OtherTok{\textless{}{-}} \FunctionTok{numeric}\NormalTok{(n\_bootstrap)}
\FunctionTok{set.seed}\NormalTok{(}\DecValTok{123}\NormalTok{)}

\ControlFlowTok{for}\NormalTok{ (i }\ControlFlowTok{in} \DecValTok{1}\SpecialCharTok{:}\NormalTok{n\_bootstrap) \{}
\NormalTok{  vaccine\_sample }\OtherTok{\textless{}{-}} \FunctionTok{sample}\NormalTok{(}\FunctionTok{c}\NormalTok{(}\DecValTok{0}\NormalTok{, }\DecValTok{1}\NormalTok{), }\AttributeTok{size =}\NormalTok{ vaccine}\SpecialCharTok{$}\NormalTok{COVID }\SpecialCharTok{+}\NormalTok{ vaccine}\SpecialCharTok{$}\NormalTok{No\_COVID, }\AttributeTok{replace =} \ConstantTok{TRUE}\NormalTok{, }\AttributeTok{prob =} \FunctionTok{c}\NormalTok{(}\DecValTok{1} \SpecialCharTok{{-}}\NormalTok{ prop\_vaccine, prop\_vaccine))}
\NormalTok{  placebo\_sample }\OtherTok{\textless{}{-}} \FunctionTok{sample}\NormalTok{(}\FunctionTok{c}\NormalTok{(}\DecValTok{0}\NormalTok{, }\DecValTok{1}\NormalTok{), }\AttributeTok{size =}\NormalTok{ placebo}\SpecialCharTok{$}\NormalTok{COVID }\SpecialCharTok{+}\NormalTok{ placebo}\SpecialCharTok{$}\NormalTok{No\_COVID, }\AttributeTok{replace =} \ConstantTok{TRUE}\NormalTok{, }\AttributeTok{prob =} \FunctionTok{c}\NormalTok{(}\DecValTok{1} \SpecialCharTok{{-}}\NormalTok{ prop\_placebo, prop\_placebo))}
  
\NormalTok{  prop\_vaccine\_boot }\OtherTok{\textless{}{-}} \FunctionTok{mean}\NormalTok{(vaccine\_sample)}
\NormalTok{  prop\_placebo\_boot }\OtherTok{\textless{}{-}} \FunctionTok{mean}\NormalTok{(placebo\_sample)}
  
\NormalTok{  bootstrap\_pi }\OtherTok{\textless{}{-}}\NormalTok{ prop\_vaccine\_boot }\SpecialCharTok{/}\NormalTok{ (prop\_vaccine\_boot }\SpecialCharTok{+}\NormalTok{ prop\_placebo\_boot)}
  
\NormalTok{  bootstrap\_psis[i] }\OtherTok{\textless{}{-}}\NormalTok{ (}\DecValTok{1} \SpecialCharTok{{-}} \DecValTok{2} \SpecialCharTok{*}\NormalTok{ bootstrap\_pi) }\SpecialCharTok{/}\NormalTok{ (}\DecValTok{1} \SpecialCharTok{{-}}\NormalTok{ bootstrap\_pi)}
\NormalTok{\}}
\end{Highlighting}
\end{Shaded}

\begin{Shaded}
\begin{Highlighting}[]
\NormalTok{bootstrap\_df }\OtherTok{\textless{}{-}} \FunctionTok{data.frame}\NormalTok{(}\AttributeTok{psi =}\NormalTok{ bootstrap\_psis)}

\NormalTok{bootstrap\_summary }\OtherTok{\textless{}{-}}\NormalTok{ bootstrap\_df }\SpecialCharTok{\%\textgreater{}\%}
  \FunctionTok{summarise}\NormalTok{(}\AttributeTok{n =} \FunctionTok{n}\NormalTok{(), }
            \AttributeTok{mean =} \FunctionTok{mean}\NormalTok{(psi), }
            \AttributeTok{sd =} \FunctionTok{sd}\NormalTok{(psi), }
            \AttributeTok{lower\_critical\_value =}\NormalTok{ mean }\SpecialCharTok{{-}} \FunctionTok{qnorm}\NormalTok{(}\FloatTok{0.975}\NormalTok{)}\SpecialCharTok{*}\NormalTok{sd, }
            \AttributeTok{upper\_critical\_value =}\NormalTok{  mean }\SpecialCharTok{+} \FunctionTok{qnorm}\NormalTok{(}\FloatTok{0.975}\NormalTok{)}\SpecialCharTok{*}\NormalTok{sd)}

\NormalTok{lower\_critical\_value }\OtherTok{\textless{}{-}}\NormalTok{ bootstrap\_summary}\SpecialCharTok{$}\NormalTok{lower\_critical\_value}
\NormalTok{upper\_critical\_value }\OtherTok{\textless{}{-}}\NormalTok{ bootstrap\_summary}\SpecialCharTok{$}\NormalTok{upper\_critical\_value}

\NormalTok{bootstrap\_summary}
\end{Highlighting}
\end{Shaded}

\begin{verbatim}
##       n      mean         sd lower_critical_value upper_critical_value
## 1 10000 0.9500209 0.01801763             0.914707            0.9853348
\end{verbatim}

\begin{Shaded}
\begin{Highlighting}[]
\FunctionTok{ggplot}\NormalTok{(bootstrap\_df, }\FunctionTok{aes}\NormalTok{(}\AttributeTok{x =}\NormalTok{ psi)) }\SpecialCharTok{+}
  \FunctionTok{geom\_histogram}\NormalTok{(}\FunctionTok{aes}\NormalTok{(}\AttributeTok{y =} \FunctionTok{after\_stat}\NormalTok{(density)), }
                 \AttributeTok{bins =} \FunctionTok{round}\NormalTok{(}\FunctionTok{log}\NormalTok{(}\FunctionTok{length}\NormalTok{(bootstrap\_df}\SpecialCharTok{$}\NormalTok{psi), }\AttributeTok{base =} \DecValTok{2}\NormalTok{)), }
                 \AttributeTok{fill =} \StringTok{"mediumpurple"}\NormalTok{, }
                 \AttributeTok{color =} \StringTok{"black"}\NormalTok{) }\SpecialCharTok{+}
  \FunctionTok{labs}\NormalTok{(}\AttributeTok{title =} \StringTok{"Histogram of Bootstrap Psi Values"}\NormalTok{, }\AttributeTok{x =} \StringTok{"Psi"}\NormalTok{) }\SpecialCharTok{+}
  \FunctionTok{theme\_minimal}\NormalTok{() }\SpecialCharTok{+}
  \FunctionTok{geom\_vline}\NormalTok{(}\AttributeTok{xintercept =}\NormalTok{ observed\_psi, }
             \AttributeTok{linetype =} \StringTok{"solid"}\NormalTok{, }
             \AttributeTok{color =} \StringTok{"red"}\NormalTok{) }\SpecialCharTok{+} 
  \FunctionTok{geom\_vline}\NormalTok{(}\AttributeTok{xintercept =}\NormalTok{ bootstrap\_summary}\SpecialCharTok{$}\NormalTok{upper\_critical\_value, }
             \AttributeTok{linetype =} \StringTok{"dashed"}\NormalTok{, }
             \AttributeTok{color =} \StringTok{"red"}\NormalTok{) }\SpecialCharTok{+} 
  \FunctionTok{geom\_vline}\NormalTok{(}\AttributeTok{xintercept =}\NormalTok{ bootstrap\_summary}\SpecialCharTok{$}\NormalTok{lower\_critical\_value, }
             \AttributeTok{linetype =} \StringTok{"dashed"}\NormalTok{, }
             \AttributeTok{color =} \StringTok{"red"}\NormalTok{) }\SpecialCharTok{+} 
  \FunctionTok{scale\_y\_continuous}\NormalTok{(}\AttributeTok{labels =}\NormalTok{ scales}\SpecialCharTok{::}\FunctionTok{number\_format}\NormalTok{(}\AttributeTok{accuracy =} \FloatTok{0.1}\NormalTok{)) }\SpecialCharTok{+}
  \FunctionTok{theme}\NormalTok{(}\AttributeTok{plot.title =} \FunctionTok{element\_text}\NormalTok{(}\AttributeTok{hjust =} \FloatTok{0.5}\NormalTok{)) }\SpecialCharTok{+}
  \FunctionTok{stat\_function}\NormalTok{(}\AttributeTok{fun =}\NormalTok{ dnorm, }\AttributeTok{args =} \FunctionTok{list}\NormalTok{(}\AttributeTok{mean =}\NormalTok{ bootstrap\_summary}\SpecialCharTok{$}\NormalTok{mean, }\AttributeTok{sd =}\NormalTok{ bootstrap\_summary}\SpecialCharTok{$}\NormalTok{sd), }\AttributeTok{color =} \StringTok{"blue"}\NormalTok{, }\AttributeTok{size =} \DecValTok{1}\NormalTok{)}
\end{Highlighting}
\end{Shaded}

\includegraphics{Writeup_files/figure-latex/unnamed-chunk-5-1.pdf}

\begin{Shaded}
\begin{Highlighting}[]
\NormalTok{ci\_data }\OtherTok{\textless{}{-}} \FunctionTok{data.frame}\NormalTok{(}
  \AttributeTok{Iteration =} \DecValTok{1}\SpecialCharTok{:}\NormalTok{n\_bootstrap,}
  \AttributeTok{Lower =} \FunctionTok{numeric}\NormalTok{(n\_bootstrap),}
  \AttributeTok{Upper =} \FunctionTok{numeric}\NormalTok{(n\_bootstrap)}
\NormalTok{)}

\ControlFlowTok{for}\NormalTok{ (i }\ControlFlowTok{in} \DecValTok{1}\SpecialCharTok{:}\NormalTok{n\_bootstrap) \{}
\NormalTok{  sample\_psis }\OtherTok{\textless{}{-}} \FunctionTok{sample}\NormalTok{(bootstrap\_psis, n\_bootstrap, }\AttributeTok{replace =} \ConstantTok{TRUE}\NormalTok{)}
\NormalTok{  ci\_data}\SpecialCharTok{$}\NormalTok{Lower[i] }\OtherTok{\textless{}{-}} \FunctionTok{mean}\NormalTok{(sample\_psis) }\SpecialCharTok{{-}} \FunctionTok{qnorm}\NormalTok{(}\FloatTok{0.975}\NormalTok{)}\SpecialCharTok{*}\FunctionTok{sd}\NormalTok{(sample\_psis)}
\NormalTok{  ci\_data}\SpecialCharTok{$}\NormalTok{Upper[i] }\OtherTok{\textless{}{-}} \FunctionTok{mean}\NormalTok{(sample\_psis) }\SpecialCharTok{+} \FunctionTok{qnorm}\NormalTok{(}\FloatTok{0.975}\NormalTok{)}\SpecialCharTok{*}\FunctionTok{sd}\NormalTok{(sample\_psis)}
\NormalTok{  ci\_data}\SpecialCharTok{$}\NormalTok{Mean[i] }\OtherTok{\textless{}{-}} \FunctionTok{mean}\NormalTok{(sample\_psis)}
\NormalTok{\}}
\end{Highlighting}
\end{Shaded}

\begin{Shaded}
\begin{Highlighting}[]
\NormalTok{plot\_data }\OtherTok{\textless{}{-}}\NormalTok{ ci\_data[}\FunctionTok{seq}\NormalTok{(}\DecValTok{1}\NormalTok{, n\_bootstrap, }\AttributeTok{by =} \DecValTok{150}\NormalTok{), ]}

\FunctionTok{ggplot}\NormalTok{(plot\_data, }\FunctionTok{aes}\NormalTok{(}\AttributeTok{y =}\NormalTok{ Iteration)) }\SpecialCharTok{+}
  \FunctionTok{geom\_segment}\NormalTok{(}\FunctionTok{aes}\NormalTok{(}\AttributeTok{yend =}\NormalTok{ Iteration, }\AttributeTok{x =}\NormalTok{ Lower, }\AttributeTok{xend =}\NormalTok{ Upper), }\AttributeTok{color =} \StringTok{"navy"}\NormalTok{) }\SpecialCharTok{+}
  \FunctionTok{geom\_point}\NormalTok{(}\FunctionTok{aes}\NormalTok{(}\AttributeTok{y =}\NormalTok{ Iteration, }\AttributeTok{x =}\NormalTok{ Mean), }\AttributeTok{color =} \StringTok{"navy"}\NormalTok{)}\SpecialCharTok{+}
  \FunctionTok{geom\_vline}\NormalTok{(}\AttributeTok{xintercept =}\NormalTok{ lower\_critical\_value, }\AttributeTok{linetype =} \StringTok{"dashed"}\NormalTok{, }\AttributeTok{color =} \StringTok{"red"}\NormalTok{) }\SpecialCharTok{+}
  \FunctionTok{geom\_vline}\NormalTok{(}\AttributeTok{xintercept =}\NormalTok{ upper\_critical\_value, }\AttributeTok{linetype =} \StringTok{"dashed"}\NormalTok{, }\AttributeTok{color =} \StringTok{"red"}\NormalTok{) }\SpecialCharTok{+}
  \FunctionTok{geom\_text}\NormalTok{(}\FunctionTok{aes}\NormalTok{(}\AttributeTok{x =}\NormalTok{ lower\_critical\_value }\SpecialCharTok{+} \FloatTok{0.012}\NormalTok{, }\AttributeTok{y =} \FunctionTok{max}\NormalTok{(Iteration) }\SpecialCharTok{+} \DecValTok{300}\NormalTok{, }\AttributeTok{label =} \StringTok{"Lower Bound"}\NormalTok{), }\AttributeTok{color =} \StringTok{"red"}\NormalTok{, }\AttributeTok{hjust =} \FloatTok{1.1}\NormalTok{) }\SpecialCharTok{+}
  \FunctionTok{geom\_text}\NormalTok{(}\FunctionTok{aes}\NormalTok{(}\AttributeTok{x =}\NormalTok{ upper\_critical\_value }\SpecialCharTok{{-}} \FloatTok{0.012}\NormalTok{, }\AttributeTok{y =} \FunctionTok{max}\NormalTok{(Iteration) }\SpecialCharTok{+} \DecValTok{300}\NormalTok{, }\AttributeTok{label =} \StringTok{"Upper Bound"}\NormalTok{), }\AttributeTok{color =} \StringTok{"red"}\NormalTok{, }\AttributeTok{hjust =} \SpecialCharTok{{-}}\FloatTok{0.1}\NormalTok{) }\SpecialCharTok{+}
  \FunctionTok{geom\_vline}\NormalTok{(}\AttributeTok{xintercept =}\NormalTok{ bootstrap\_summary}\SpecialCharTok{$}\NormalTok{mean, }\AttributeTok{linetype =} \StringTok{"dashed"}\NormalTok{, }\AttributeTok{color =} \StringTok{"red"}\NormalTok{) }\SpecialCharTok{+} 
  \FunctionTok{labs}\NormalTok{(}\AttributeTok{title =} \StringTok{"Confidence Intervals of Bootstrap Samples"}\NormalTok{,}
       \AttributeTok{y =} \StringTok{"Iteration"}\NormalTok{,}
       \AttributeTok{x =} \StringTok{"Confidence Interval"}\NormalTok{,}
       \AttributeTok{subtitle =} \StringTok{"Each line segment represents a 95\% confidence interval for a bootstrapped sample"}\NormalTok{) }\SpecialCharTok{+}
  \FunctionTok{theme\_minimal}\NormalTok{() }\SpecialCharTok{+}
  \FunctionTok{theme}\NormalTok{(}\AttributeTok{plot.title =} \FunctionTok{element\_text}\NormalTok{(}\AttributeTok{hjust =} \FloatTok{0.5}\NormalTok{),}
        \AttributeTok{plot.subtitle =} \FunctionTok{element\_text}\NormalTok{(}\AttributeTok{hjust =} \FloatTok{0.5}\NormalTok{, }\AttributeTok{size =} \DecValTok{10}\NormalTok{),}
        \AttributeTok{axis.text.y =} \FunctionTok{element\_text}\NormalTok{(}\AttributeTok{hjust =} \DecValTok{1}\NormalTok{))}
\end{Highlighting}
\end{Shaded}

\includegraphics{Writeup_files/figure-latex/unnamed-chunk-7-1.pdf}

\begin{Shaded}
\begin{Highlighting}[]
\NormalTok{p\_val }\OtherTok{\textless{}{-}} \FunctionTok{pnorm}\NormalTok{(}\FloatTok{0.3}\NormalTok{, }\AttributeTok{mean =}\NormalTok{ bootstrap\_summary}\SpecialCharTok{$}\NormalTok{mean, }\AttributeTok{sd =}\NormalTok{ bootstrap\_summary}\SpecialCharTok{$}\NormalTok{sd, }\AttributeTok{lower.tail =}\NormalTok{ T)}

\NormalTok{emp\_p\_val }\OtherTok{\textless{}{-}} \FunctionTok{length}\NormalTok{(bootstrap\_df[bootstrap\_df}\SpecialCharTok{$}\NormalTok{psi }\SpecialCharTok{\textless{}=} \FloatTok{0.3}\NormalTok{])}\SpecialCharTok{/}\FunctionTok{length}\NormalTok{(bootstrap\_df}\SpecialCharTok{$}\NormalTok{psi)}

\FunctionTok{cat}\NormalTok{(}\StringTok{"Estimated p{-}value:"}\NormalTok{, emp\_p\_val, }\StringTok{"}\SpecialCharTok{\textbackslash{}n}\StringTok{"}\NormalTok{)}
\end{Highlighting}
\end{Shaded}

\begin{verbatim}
## Estimated p-value: 0
\end{verbatim}

\begin{Shaded}
\begin{Highlighting}[]
\FunctionTok{cat}\NormalTok{(}\StringTok{"Empirical p{-}value:"}\NormalTok{, emp\_p\_val, }\StringTok{"}\SpecialCharTok{\textbackslash{}n}\StringTok{"}\NormalTok{)}
\end{Highlighting}
\end{Shaded}

\begin{verbatim}
## Empirical p-value: 0
\end{verbatim}

\end{document}
